\documentclass[a4paper,12pt]{article}
\usepackage[left=1.5cm, right=1.5cm, top=2cm,bottom=2cm]{geometry}
\usepackage{amsmath, amssymb, amsfonts}
\usepackage{multicol}
\usepackage{enumerate}
\usepackage{xcolor}

% https://tex.stackexchange.com/questions/29834/closed-square-root-symbol
\usepackage{letltxmacro}
\makeatletter
\let\oldr@@t\r@@t
\def\r@@t#1#2{%
\setbox0=\hbox{$\oldr@@t#1{#2\,}$}\dimen0=\ht0
\advance\dimen0-0.2\ht0
\setbox2=\hbox{\vrule height\ht0 depth -\dimen0}%
{\box0\lower0.4pt\box2}}
\LetLtxMacro{\oldsqrt}{\sqrt}
\renewcommand*{\sqrt}[2][\ ]{\oldsqrt[#1]{#2}}
\makeatother

\begin{document} 
  
\section*{Piape Matemática} 
\textbf{Módulo II - Malabarismos Algébricos}\\
\textbf{Exercícios Aula 05}         
\begin{multicols}{2}
\paragraph*{1.} Calcule a soma dos seguintes polinômios:
\begin{enumerate}[a)]
\item $(2x^2 - 3x + 1) + (3x^2 + 2x - 1)$
\item $(3x^3 - 2x^2 + 4x - 1) + (2x^3 + 3x^2 - 2x + 1)$
\item $(x^4 - 2x^3 + 3x^2 - 4x + 5) + ( 3x^3 - 4x^2 + 5x - 6)$ 
\end{enumerate}
\paragraph*{2.} Calcule o produto dos seguintes polinômios:
\begin{enumerate}[a)]
\item $(2x - 3)(3x + 2)$
\item $(3x - 2)(2x + 1)$
\item $(x^2 - 2x + 1)(2x + 1)$
\item $(x^3 - 2x^2 + 1)(x^3 - 1)$
\end{enumerate}

\paragraph*{3.} Calcule as divisões de polinômios:
\begin{enumerate}[a)]
\item $(2x^2 - 3x + 1) \div (x - 1)$
\item $(3x^3 - 2x^2 + 4x - 1) \div (x - 1)$
\item $(x^4  + 3x^2   + 5) \div (x^2 - 1)$
\item $(x^4  + 2x^2   + 5x) \div (x^2 + 1)$
\item $(3x^3-4x^2-x+2)\div(3x+2)$
\end{enumerate}

\paragraph*{4.} Nos items abaixo, são dados um polinômio e um par de números reais. Um desses números é uma raiz do polinômio. Utilize essa raiz para fatorar o polinômio.
\begin{enumerate}[a)]
\item $2x^2 + x - 3$; números: -1 e 1;
\item $-x^2 + 4x - 3$; números: 3 e 4; 
\item $x^3 + x^2 + 2x + 2$; números: -1 e 2;
\item $x^3 + 2x^2 -4x + 8$; números: 2 e 4;
\end{enumerate}

\vspace*{5cm}
\end{multicols}
 
\vspace*{\fill}
{\footnotesize \color{darkgray}
\paragraph*{Gabarito} \hspace*{\fill}\\
\textbf{1.} a) $5x^2 - x$; b) $5x^3 + x^2 + 2x$; c) $x^4 + x^3 - x^2 + x - 1$\\
\textbf{2.} a) $6x^2 + x - 6$; b) $6x^2 - x - 2$; c) $2x^3 - 3x^2 + 1$; d) $x^6 - 3x^5 + 1$ \\
\textbf{3.} a) $2x - 1$ com resto 0; b) $3x^2 + x + 5$ com resto 4; c) $x^2 + 4$ com resto 9; d) $x^2  + 1$ com resto $5x-1$; e) $x^2-2x+1$ com resto 0.\\
\textbf{4.} a) raiz 1 e fatoração $(x-1)(2x+3)$; b) raiz 3 e fatoração $(-x+3)(x-1)$; c) raiz -1 e fatoração $(x+1)(x^2+2)$ d) raiz 2 e fatoração $(x-2)(x^2+4x+4)$.
}
\end{document}