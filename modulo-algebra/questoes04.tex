\documentclass[a4paper,12pt]{article}
\usepackage[left=1.5cm, right=1.5cm, top=2cm,bottom=2cm]{geometry}
\usepackage{amsmath, amssymb, amsfonts}
\usepackage{multicol}
\usepackage{enumerate}
\begin{document} 
  
\section*{Piape Matemática} 
\textbf{Módulo II - Malabarismos Algébricos}\\
\textbf{Exercícios Aula 04}         
\begin{multicols}{2}
\paragraph{1.} Simplifique os radicais:
\begin{enumerate}[a)]
    \item $\sqrt{28}$ 
    \item $\sqrt{18}$
    \item $\sqrt[3]{108}$
    \item $\sqrt[4]{144}$
    \item $\sqrt{16\beta^2}$
    \item $\sqrt[3]{54\omega^6\lambda^4}$
    \item $\sqrt{\frac{4}{9}b^3}$ 
\end{enumerate} 

\paragraph*{2.} Racionalize os denominadores:
\begin{enumerate}[a)]
    \item $\displaystyle\frac{4}{\sqrt{3}}$
    \item $\displaystyle\frac{\sqrt[5]{6}}{\sqrt{5}}$
    \item $\displaystyle\frac{-3}{2+\sqrt{7}}$
    \item $\displaystyle\frac{1+\sqrt{2}}{3-\sqrt{3}}$
    \item $\displaystyle\frac{1-\sqrt{2}}{4-\sqrt{5}}$
\end{enumerate}
 
\paragraph*{3.} Transforme para o mesmo índice do radical, utilizando expoentes fracionários:
\begin{enumerate}[a)]
    \item $\sqrt[3]{2}$ e $\sqrt{7}$
    \item $\sqrt[4]{{3^3}}$ e $\sqrt{5}$
    \item $\sqrt[5]{6^3}$ e $\sqrt[3]{4}$
\end{enumerate}

\paragraph*{4.} Simplifique as expressões:
\begin{enumerate}[a)]
    \item $\sqrt{7}\cdot\sqrt{14}$
    \item $\sqrt[3]{24}\cdot\sqrt[3]{18}$
    \item $\sqrt{3}\cdot\sqrt[3]{12}$
    \item $\displaystyle\frac{1}{\sqrt{2}} + \frac{1}{\sqrt{8}}$
\end{enumerate}
\end{multicols}
 
\vspace*{\fill}
{\footnotesize
\paragraph*{Gabarito} \hspace*{\fill}\\
\textbf{1.} \\
}
\end{document}