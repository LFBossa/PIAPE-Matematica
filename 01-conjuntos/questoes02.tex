\documentclass[a4paper,twocolumn,12pt]{article}
\usepackage[left=1.5cm, right=1.5cm, top=2cm,bottom=2cm]{geometry}
\usepackage{amsmath, amssymb, amsfonts}
\usepackage{enumerate}
\begin{document} 
\section*{Piape Matemática} 
 
\subsection*{Módulo I}
 
\subsection*{Exercícios Aula 03}

\paragraph{1.} Para os exercício que segue, considere os seguintes conjuntos:
\begin{align*}
P &= \{1,2,3,4,8\}\\
Q &=  \{2,4,6\}\\
R &= \{1,3,5\}
\end{align*}
%\[ P = \{1,2,3,4,8\}\] \[ Q =  \{2,4,6\}\] \[ R = \{1,3,5\}\]
Calcule o que se pede. Represente o resultado em notação de diagramas de Venn.

\medskip

\begin{minipage}[t]{0.45\columnwidth}
  \begin{enumerate}[a)]
    \item \(P\cup Q\)
    \item \(Q\cup R\)
    \item \(P\cup Q \cup R\)
    \item \(P\cap Q\)
    \item \(Q\cap R\)
  \end{enumerate}
\end{minipage}\begin{minipage}[t]{0.45\columnwidth}
  \begin{enumerate}[a)]
    \setcounter{enumi}{5}
    \item \(P\setminus Q\)
    \item \(Q\setminus P\)
    \item \(P\setminus R\)
    \item \(R\cap P\)
  \end{enumerate}
\end{minipage} 

\paragraph{2.} Calcule os tamanhos dos conjuntos
\begin{enumerate}[a)]
  \item \(|P\cup Q|\)
  \item \(|P\cap Q|\)
  \item \(|Q\cup R|\)
  \item \(|Q\cap R|\)
\end{enumerate}

Verifique que em todos os casos, vale a relação
\[|X\cup Y| = |X| + |Y| - |X\cap Y|\]


\paragraph{3.} Em uma turma, as pessoas ou praticam natação ou praticam vôlei. Sabe-se que 20 pessoas praticam natação, 15 praticam vôlei e 5 praticam ambos. Quantas pessoas há na turma?


\paragraph{4.} Uma turma possui 60 alunos. Destes, 40 praticam natação, 30 praticam vôlei e 20 praticam ambos. Quantos alunos não praticam nenhuma das duas atividades?

\paragraph{5.} Luciana está em um supermercado representando uma marca de café e, a cada cliente que aborda, ela oferece uma amostra grátis de três tipos de café (X, Y e Z). Após a degustação, o cliente responde a uma enquete a respeito dos tipos de café que gostou. Após coletar as respostas de 400 clientes, ela concluiu que:
\begin{itemize}
  \setlength{\itemsep}{1pt}
  \setlength{\itemindent}{1pt}
  \item 184 clientes gostaram do tipo X;
  \item 188 clientes gostaram do tipo Y;
  \item 220 clientes gostaram do tipo Z;
  \item 76 clientes gostaram dos tipos X e Y;
  \item 84 clientes gostaram dos tipos X e Z;
  \item 120 clientes gostaram dos tipos Y e Z; e,
  \item 52 clientes não gostaram de nenhum dos tipos X, Y e Z.
\end{itemize}
Considerando as informações, qual o número de clientes que gostaram dos três tipos de café? Utilize a fórmula 
\begin{multline*}
|P\cup Q\cup R| = |P| + |Q| + |R| \\ - |P\cap Q| - |Q\cap R| - |R\cap P| \\ + |P\cap Q \cap R|
\end{multline*} 


\paragraph{6.} Essa é uma questão mais teórica: 
\begin{enumerate}[a)] 
\item Qual relação de inclusão entre os conjuntos \(P\), \(Q\) e \(P\cup Q\)?
\item Qual relação de inclusão entre os conjuntos \(P\), \(Q\) e \(P\cap Q\)?
\end{enumerate}


\vfill

{\footnotesize
\paragraph*{Gabarito} \hspace*{\fill}\\
\textbf{1.} a) \(P\cup Q = \{1,2,3,4,6,8\}\); b) \(Q\cup R = \{1,2,3,4,5,6\}\); c) \(P\cup Q \cup R = \{1,2,3,4,5,6,8\}\); d) \(P\cap Q = \{2,4\}\); e) \(Q\cap R = \emptyset\); f) \(P\setminus Q = \{1,3,8\}\); g) \(Q\setminus P = \{6\}\); h) \(P\setminus R = \{2,4,8\}\); i) \(R\cap P = \{1,3\}\)\\
\textbf{2.} a) \(|P\cup Q| = 6\); b) \(|P\cap Q| = 2\); c) \(|Q\cup R| = 6\); d) \(|Q\cap R| = 0\)\\
\textbf{3.} 30 pessoas; \textbf{4.} 10 alunos; \textbf{5.} 36 clientes\\
\textbf{6.} a) \(P\subseteq P\cup Q\) e também $Q\subseteq P\cup Q$; b) \(P\cap Q\subseteq P\) e também \(P\cap Q\subseteq Q\)
}

\end{document}