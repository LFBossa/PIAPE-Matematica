\documentclass[a4paper,twocolumn,12pt]{article}
\usepackage[left=1.5cm, right=1.5cm, top=2cm,bottom=2cm]{geometry}
\usepackage{amsmath, amssymb, amsfonts}
\usepackage{enumerate}
\begin{document} 
\section*{Piape Matemática} 
 
\subsection*{Módulo I}
\subsection*{Exercícios Aula 04}

\paragraph{1.} Classifique as afirmações em verdadeiras ou falsas:
\begin{enumerate}[a)]
\item Todo número natural também é inteiro. 
\item Todo número inteiro também é natural.  
\item Existem números inteiros que não são naturais.
\item Todo número racional é inteiro. 
\item Todo número inteiro é racional.
\item $\sqrt{2}$ é um número racional.
\item $\sqrt{2} \notin \mathbb{Q}$.
\item $\sqrt{2} \in \mathbb{R\backslash Q}$.
\end{enumerate}

\paragraph*{2. } Vamos trabalhar em $\mathbb{Q}$, o conjunto dos números racionais. Calcule o que se pede:

\begin{enumerate}[a)]
  \item $\displaystyle\frac{1}{3} + \frac{2}{5}$
  \item $\displaystyle\frac{4}{7} - \frac{1}{3}$
  \item $\displaystyle\frac{3}{4} + \frac{1}{2} - \frac{1}{8}$
  \item $\displaystyle\frac{2}{7}\cdot \frac{3}{5}$
  \item $\displaystyle-\frac{3}{4} \cdot \frac{1}{2}$
  \item $\displaystyle\frac{3}{4} \div \frac{2}{3}$
  \item $\displaystyle \frac{7}{3} \div 5$
\end{enumerate}


\paragraph*{3. } Converta as frações em números decimais:
\begin{enumerate}[a)]
  \item $\displaystyle-\frac{1}{7}$
  \item $\displaystyle\frac{2}{5}$
  \item $\displaystyle\frac{3}{4}$
  \item $\displaystyle-\frac{5}{8}$
\end{enumerate}

\paragraph*{4. } Converta os números decimais em frações:
\begin{enumerate}[a)]
  \item $3,14$
  \item $3,1411\bar{1}$
  \item $0,3333...$
  \item $-2,5$
  \item $-0,75$
\end{enumerate}


\paragraph*{5. } Racionalize o denominador das seguintes frações:
\begin{enumerate}[a)]
  \item $\displaystyle\frac{1}{\sqrt{2}}$
  \item $\displaystyle\frac{2}{\sqrt{3}}$
  \item $\displaystyle\frac{-3}{\sqrt{5}}$
  \item $\displaystyle\frac{2\sqrt{2}}{\sqrt{7}}$
  \item $\displaystyle\frac{\sqrt{3}}{\sqrt{11}}$
  \end{enumerate}

\paragraph*{6. } As afirmações da questão 1 podem ser traduzidas em símbolos matemáticos. Associe os símbolos abaixo com as afirmações correspondentes:
\begin{enumerate}[a)]
  \item (\hspace{7mm}) $\sqrt{2} \in \mathbb{Q}$ 
  \item (\hspace{7mm}) $\mathbb{Z} \subseteq \mathbb{Q}$
  \item (\hspace{7mm}) $\mathbb{Q} \subseteq \mathbb{Z}$
  \item (\hspace{7mm}) $\mathbb{Z} \subseteq \mathbb{N}$
  \item (\hspace{7mm}) $\mathbb{N} \subseteq \mathbb{Z}$
\end{enumerate}
\end{document}