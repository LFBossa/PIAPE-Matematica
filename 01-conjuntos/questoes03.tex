\documentclass[a4paper,twocolumn,12pt]{article}
\usepackage[left=1.5cm, right=1.5cm, top=2cm,bottom=2cm]{geometry}
\usepackage{amsmath, amssymb, amsfonts}
\usepackage{enumerate}
\begin{document} 
\section*{Piape Matemática} 
 
\subsection*{Módulo I}
\subsection*{Exercícios Aula 04}

\paragraph{1.} Classifique as afirmações em verdadeiras ou falsas:
\begin{enumerate}[a)]
\item Todo número natural também é inteiro. 
\item Todo número inteiro também é natural.  
\item Existem números inteiros que não são naturais.
\item Todo número racional é inteiro. 
\item Todo número inteiro é racional.
\item $\sqrt{2}$ é um número racional.
\item $\sqrt{2} \notin \mathbb{Q}$.
\item $\sqrt{2} \in \mathbb{R\backslash Q}$.
\end{enumerate}


\paragraph*{2. } As afirmações da questão 1 podem ser traduzidas em símbolos matemáticos. Associe os símbolos abaixo com as afirmações correspondentes:
\begin{enumerate}[a)]
  \item (\hspace{7mm}) $\sqrt{2} \in \mathbb{Q}$ 
  \item (\hspace{7mm}) $\mathbb{Z} \subseteq \mathbb{Q}$
  \item (\hspace{7mm}) $\mathbb{Q} \subseteq \mathbb{Z}$
  \item (\hspace{7mm}) $\mathbb{Z} \subseteq \mathbb{N}$
  \item (\hspace{7mm}) $\mathbb{N} \subseteq \mathbb{Z}$
\end{enumerate}


\vfill
% REMOVER OPERAÇÕES COM FRAÇÕES

{\footnotesize
\paragraph*{Gabarito} \hspace*{\fill}\\
\textbf{1.} a) Verdadeira; b) Falsa; c) Verdadeira; d) Falsa; e) Verdadeira; f) Falsa; g) Verdadeira; h) Verdadeira\\ 
\textbf{2.} a) f); b) e); c) b); d) c); e) d)\\
}
\end{document}