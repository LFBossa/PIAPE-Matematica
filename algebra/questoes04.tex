\documentclass[a4paper,12pt]{article}
\usepackage[left=1.5cm, right=1.5cm, top=2cm,bottom=2cm]{geometry}
\usepackage{amsmath, amssymb, amsfonts}
\usepackage{multicol}
\usepackage{enumerate}

% https://tex.stackexchange.com/questions/29834/closed-square-root-symbol
\usepackage{letltxmacro}
\makeatletter
\let\oldr@@t\r@@t
\def\r@@t#1#2{%
\setbox0=\hbox{$\oldr@@t#1{#2\,}$}\dimen0=\ht0
\advance\dimen0-0.2\ht0
\setbox2=\hbox{\vrule height\ht0 depth -\dimen0}%
{\box0\lower0.4pt\box2}}
\LetLtxMacro{\oldsqrt}{\sqrt}
\renewcommand*{\sqrt}[2][\ ]{\oldsqrt[#1]{#2}}
\makeatother

\begin{document} 
  
\section*{Piape Matemática} 
\textbf{Módulo II - Malabarismos Algébricos}\\
\textbf{Exercícios Aula 04}         
\begin{multicols}{2}
\paragraph{1.} Simplifique os radicais:
\begin{enumerate}[a)]
    \item $\sqrt{28}$ 
    \item $\sqrt{18}$
    \item $\sqrt[3]{108}$
    \item $\sqrt[4]{144}$
    \item $\sqrt{16\beta^2}$
    \item $\sqrt[3]{54\omega^6\lambda^4}$
    \item $\displaystyle\sqrt{\frac{4}{9}b^3}$ 
\end{enumerate} 

\paragraph*{2.} Racionalize os denominadores:
\begin{enumerate}[a)]
    \item $\displaystyle\frac{4}{\sqrt{3}}$
    \item $\displaystyle\frac{\sqrt[5]{6}}{\sqrt{5}}$
    \item $\displaystyle\frac{-3}{2+\sqrt{7}}$
    \item $\displaystyle\frac{1+\sqrt{2}}{3-\sqrt{3}}$
    \item $\displaystyle\frac{1-\sqrt{2}}{4+\sqrt{5}}$
    \item $\displaystyle\frac{\sqrt[3]{8}}{3-\sqrt{8}}$
\end{enumerate}
 
\paragraph*{3.} Transforme para o mesmo índice do radical, utilizando expoentes fracionários:
\begin{enumerate}[a)]
    \item $\sqrt[3]{2}$ e $\sqrt{7}$
    \item $\sqrt[4]{{3^3}}$ e $\sqrt{5}$
    \item $\sqrt[5]{6^3}$ e $\sqrt[3]{4}$
\end{enumerate}

\paragraph*{4.} Simplifique as expressões:
\begin{enumerate}[a)]
    \item $\sqrt{7}\cdot\sqrt{14}$
    \item $\sqrt[3]{24}\cdot\sqrt[3]{18}$
    \item $\sqrt{12}\cdot\sqrt[3]{3}$
    \item $\displaystyle\frac{1}{\sqrt{2}}\cdot \frac{1}{\sqrt{8}}$
\end{enumerate}

\paragraph*{5.} Resolva as equações, utilizando a propriedade do módulo:
\begin{enumerate}[a)]
    \item $x^2 = 16$
    \item $x^2 = 144$
    \item $(x+1)^2 = 16$
    \item $(x-2)^2 = 144$
    \item $(x+3)^2 = 25$
\end{enumerate}
\vspace*{3cm}
\end{multicols}
 
\vspace*{\fill}
{\footnotesize
\paragraph*{Gabarito} \hspace*{\fill}\\
\textbf{1.} a) $2\sqrt{7}$ b) $3\sqrt{2}$ c) $3\sqrt[3]{4}$ d) $2\sqrt[4]{9}$ e) $4\beta$ f) $3\omega^2\lambda\sqrt[3]{2\lambda}$ g) $\frac{2}{3}b\sqrt{b}$ \\[1ex]
% a) $4\sqrt{3}/3$ b) $\sqrt[5]{6}\sqrt{5}/5$ c) $-3(2-\sqrt{7})/3$ d) $(1+\sqrt{2})(3+\sqrt{3})/6$ e) $(1-\sqrt{2})(4-\sqrt{5})/11$ f) $\sqrt[3]{8}(3+\sqrt{8})/7$ \\
\textbf{2.}  a) $\dfrac{4\sqrt{3}}{3}$ b) $\dfrac{\sqrt[5]{6}\sqrt{5}}{5}$ c) $\dfrac{-3(2-\sqrt{7})}{3}$ d) $\dfrac{(1+\sqrt{2})(3+\sqrt{3})}{6}$ e) $\dfrac{(1-\sqrt{2})(4-\sqrt{5})}{11}$ f) $\dfrac{\sqrt[3]{8}(3+\sqrt{8})}{1}$ \\[1ex]
\textbf{3.} a) $\sqrt[6]{2^2}$ e $\sqrt[6]{7^3}$ b) $\sqrt[4]{3^3}$ e $\sqrt[4]{5^2}$ c) $\sqrt[15]{6^9}$ e $\sqrt[15]{4^5}$ \\
\textbf{4.} a) $7\sqrt{2}$ b) $6\sqrt[3]{2}$  c) $2\sqrt[6]{3^5}$  d) $\frac{1}{4}$  \\
\textbf{5.} a) $x = \pm 4$ b) $x = \pm 12$ c) $x = 3$ e $x = -5$ d) $x = 14$ e $x=-10$ e) $x = 2$ e $x = -8$
}
\end{document}