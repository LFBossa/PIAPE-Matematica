\documentclass[a4paper,12pt]{article}
\usepackage[left=1.5cm, right=1.5cm, top=2cm,bottom=1cm]{geometry}
\usepackage{amsmath, amssymb, amsfonts, amsthm}
\usepackage{multicol}
\usepackage{enumerate}
\usepackage{pgfplots}
\usepackage{tikz}

% https://tex.stackexchange.com/questions/29834/closed-square-root-symbol
\usepackage{letltxmacro}
\makeatletter
\let\oldr@@t\r@@t
\def\r@@t#1#2{%
\setbox0=\hbox{$\oldr@@t#1{#2\,}$}\dimen0=\ht0
\advance\dimen0-0.2\ht0
\setbox2=\hbox{\vrule height\ht0 depth -\dimen0}%
{\box0\lower0.4pt\box2}}
\LetLtxMacro{\oldsqrt}{\sqrt}
\renewcommand*{\sqrt}[2][\ ]{\oldsqrt[#1]{#2}}
\makeatother

\newcommand{\novografico}[3]{
\begin{tikzpicture}
    \begin{axis}[
        grid=both,
        grid style={line width=.1pt, draw=gray!50, dashed},
        major grid style={line width=.2pt,draw=gray!50, dashed},
        axis lines=middle,
        enlargelimits,
        samples=75,
        domain=#1,
        width=5cm,
        height=5cm,
        xlabel=$x$,
        ylabel={$#3$},
        xlabel style={at={(current axis.right of origin)}, anchor=west},
        ylabel style={at={(current axis.above origin)}, anchor=south},
        unbounded coords=jump
    ]
    \addplot[blue, thick] {#2}; 
    \end{axis}
\end{tikzpicture}
}

\begin{document} 
  
\section*{Piape Matemática} 
\textbf{Módulo IV - Tudo é função}\\
\textbf{Exercícios Aula 06}    

\begin{multicols}{2} 
 
\paragraph*{1.}  Descubra se as funções abaixo são pares, ímpares ou sem paridade. Caso não tenha paridade, justifique calculando o valor da função em diferentes pontos.

\begin{enumerate}[a)]
\item $f(x) = x^3$
\item $g(x) = x^3 - 2x$
\item $h(x) = x^4 + 2$
\item $i(x) = x + \dfrac{4}{x}$ 
\item $j(x) = x^2 + 2x + 1$
\end{enumerate}

\paragraph*{2.} Classifique as funções em crescente, decrescente.

\novografico{-2:2}{10/(x+3)}{f(x)}
\novografico{-2:2}{x^3}{g(x)}

\novografico{-2:2}{1/(1+exp(-2*x))}{h(x)}
\novografico{0.1:2}{ln(x)}{r(x)}

\paragraph*{3.} Questões teóricas. Vamos pensar um pouco sobre funções.
\begin{enumerate}[a)]
    \item Seja $f(x)$ uma função tal que $f(0) = \frac{3}{4}$. É possível que $f$ seja ímpar?
    \item Considere uma função $g$ qualquer que seja par. É possível que essa função seja injetora?
    \item Seja $h$ uma função crescente. Você consegue me explicar por que $h$ é necessariamente injetora? O mesmo vale se $h$ for decrescente?
    \item Seja $k$ uma função crescente. Explique por que $k$ não pode ser par. E se $k$ for decrescente?
\end{enumerate}

\paragraph*{4.} Esboçe o gráfico da função $j(x) = x^2 - 2x + 1$ da questão 1e.
\begin{enumerate}[a)]
    \item Como você restringiria o domínio de $j$ para que ela seja injetora?
    \item Como você restringiria o contradomínio de $j$ para que ela seja sobrejetora?
\end{enumerate}
 \vspace*{4cm}
\end{multicols}

\vspace*{\fill}
{\footnotesize
\paragraph*{Gabarito} \hspace*{\fill}\\
\textbf{1.} a) ímpar; b) ímpar; c) par; d) ímpar; e) sem paridade.\\   
\textbf{2.} a) decrescente; b) crescente; c) crescente; d) crescente.\\
\textbf{3.} a) Não, pois numa função ímpar, $f(0) = 0$; b)  Não, pois por exemplo, $g(-2) = g(2)$; c) Se $a\neq b$, então uma das duas coisas acontece: $a <b$ ou $a > b$. Se $a<b$ então $f(a) < f(b)$, ou seja, $f(a) \neq f(b)$. Se $a > b$, então $f(a) > f(b)$, ou seja, $f(a) \neq f(b)$. Portanto, $f$ é injetora. Se $f$ for decrescente, a mesma coisa vale. d)  Se existisse uma função $k$ que fosse crescente e par, então teríamos uma contradição: pelo item 
c), se $k$ é crescente então $k$ é injetora, porém pelo item (b) se $k$ é par então $k$ não pode ser injetora.\\
\textbf{4.} a) $j$ é injetora para $x \in [-1, \infty)$; b) $j$ é sobrejetora para $y \in [0, \infty)$.
}
\end{document}