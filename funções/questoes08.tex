\documentclass[a4paper,12pt]{article}
\usepackage[left=1.5cm, right=1.5cm, top=2cm,bottom=1cm]{geometry}
\usepackage{amsmath, amssymb, amsfonts, amsthm}
\usepackage{multicol}
\usepackage{enumerate}
\usepackage{pgfplots}
\usepackage{tikz}

% https://tex.stackexchange.com/questions/29834/closed-square-root-symbol
\usepackage{letltxmacro}
\makeatletter
\let\oldr@@t\r@@t
\def\r@@t#1#2{%
\setbox0=\hbox{$\oldr@@t#1{#2\,}$}\dimen0=\ht0
\advance\dimen0-0.2\ht0
\setbox2=\hbox{\vrule height\ht0 depth -\dimen0}%
{\box0\lower0.4pt\box2}}
\LetLtxMacro{\oldsqrt}{\sqrt}
\renewcommand*{\sqrt}[2][\ ]{\oldsqrt[#1]{#2}}
\makeatother

\newcommand{\novografico}[3]{
\begin{tikzpicture}
    \begin{axis}[
        grid=both,
        grid style={line width=.1pt, draw=gray!50, dashed},
        major grid style={line width=.2pt,draw=gray!50, dashed},
        axis lines=middle,
        enlargelimits,
        samples=75,
        domain=#1,
        width=5cm,
        height=5cm,
        xlabel=$x$,
        ylabel={$#3$},
        xlabel style={at={(current axis.right of origin)}, anchor=west},
        ylabel style={at={(current axis.above origin)}, anchor=south},
        unbounded coords=jump
    ]
    \addplot[blue, thick] {#2}; 
    \end{axis}
\end{tikzpicture}
}

\begin{document} 
  
\section*{Piape Matemática} 
\textbf{Módulo IV - Tudo é função}\\
\textbf{Exercícios Aula 08}    

\begin{multicols}{2} 
 
\paragraph*{1.}  Converta os ângulos de graus para radianos:
\begin{enumerate}[a)]
\item $36^{\circ}$
\item $20^{\circ}$
\item $120^{\circ}$
\item $135^{\circ}$
\item $72^{\circ}$
\item $144^{\circ}$
\end{enumerate}
 
\paragraph*{2.} Utilizando a fórmula da soma de arcos e os arcos notáveis, determine o valor do seno e cosseno dos seguintes ângulos:
 
Exemplo: $75^{\circ} = 30^{\circ} + 45^{\circ}$, logo
\begin{align*}
    \sin(75^{\circ}) &= \sin(30^{\circ} + 45^{\circ}) \\
    \cos(75^{\circ}) &= \cos(30^{\circ} + 45^{\circ}) 
\end{align*} 

\begin{enumerate}[a)]
    \item $75^{\circ}$
    \item $105^{\circ}$
    \item $15^{\circ}$
\end{enumerate}

\end{multicols}

\vspace*{\fill}
{\footnotesize
\paragraph*{Gabarito} \hspace*{\fill}\\
\textbf{1.} a) $\frac{\pi}{5}$, b) $\frac{\pi}{9}$, c) $\frac{2\pi}{3}$, d) $\frac{3\pi}{4}$, e) $\frac{2\pi}{5}$, f) $\frac{4\pi}{5}$\\
\textbf{2.} a) $\sin(75^{\circ}) = \frac{\sqrt{6} + \sqrt{2}}{4}$, $\cos(75^{\circ}) = \frac{\sqrt{6} - \sqrt{2}}{4}$;
b) $\sin(105^{\circ}) = \frac{\sqrt{6} - \sqrt{2}}{4}$, $\cos(105^{\circ}) = \frac{\sqrt{6} + \sqrt{2}}{4}$;
c) $\sin(15^{\circ}) = \frac{\sqrt{6} - \sqrt{2}}{4}$, $\cos(15^{\circ}) = \frac{\sqrt{6} + \sqrt{2}}{4}$
}
\end{document}