\documentclass[a4paper,12pt]{article}
\usepackage[left=1.5cm, right=1.5cm, top=2cm,bottom=2cm]{geometry}
\usepackage{amsmath, amssymb, amsfonts}
\usepackage{multicol}
\usepackage{enumerate}
\usepackage{xcolor}
% https://tex.stackexchange.com/questions/29834/closed-square-root-symbol
\usepackage{letltxmacro}
\makeatletter
\let\oldr@@t\r@@t
\def\r@@t#1#2{%
\setbox0=\hbox{$\oldr@@t#1{#2\,}$}\dimen0=\ht0
\advance\dimen0-0.2\ht0
\setbox2=\hbox{\vrule height\ht0 depth -\dimen0}%
{\box0\lower0.4pt\box2}}
\LetLtxMacro{\oldsqrt}{\sqrt}
\renewcommand*{\sqrt}[2][\ ]{\oldsqrt[#1]{#2}}
\makeatother

\begin{document} 
  
\section*{Piape Matemática} 
\textbf{Módulo II - Malabarismos Algébricos}\\
\textbf{Exercícios Aula 03}         
\begin{multicols}{2}
\paragraph{1.} Expanda os seguintes quadrados da soma
\begin{enumerate}[a)]
\item $(x + 1)^2$
\item $(2\beta + 3)^2$
\item $(\omega^2 + \sqrt{2})^2$
\item $\displaystyle\left(\frac{2}{3} + b\right)^2$
\end{enumerate} 

\paragraph{2.} Expanda os seguintes quadrados da diferença
\begin{enumerate}[a)]
\item $(x - 1)^2$
\item $(4\gamma^3 - 1)^2$
\item $(\sqrt{2}\omega - \sqrt{3})^2$
\item $\displaystyle\left(\frac{1}{3} - y\right)^2$
\end{enumerate} 

\paragraph{3.} Expanda os seguintes produtos conjugados
\begin{enumerate}[a)]
\item $(x - 1)(x + 1)$
\item $\left(2\gamma - \sqrt{7}\right)\left(2\gamma + \sqrt{7}\right)$
\item $\left(\sqrt{2}\lambda + \sqrt{3}\right)\left(\sqrt{2}\lambda - \sqrt{3}\right)$
\item $\displaystyle\left(-\frac{2}{7} + y\right)\left(\frac{2}{7} + y\right)$
\end{enumerate}  
\paragraph{4.} Fatore as seguintes expressões:
\begin{enumerate}[a)]
\item $x^2 + 6x + 9$
\item $a^2 - 2a + 1$
\item $x^4 - 2x^2 + 1$
\item $4b^2 + 8b + 4$ 
\item $a^4 - 4$
\item $4c^2 + 4\sqrt{3}c + 3$
\item $\displaystyle \frac{4}{9}z^2 - \frac{z}{3} + \frac{1}{16}$
\item $\displaystyle \frac{g^2}{4} - 4$ 
\item $\displaystyle \frac{g^2}{4} - 3$ 
\end{enumerate}
\vspace*{3cm}
\end{multicols}
 
\vspace*{\fill}
{\footnotesize \color{darkgray}
\paragraph*{Gabarito} \hspace*{\fill}\\
\textbf{1.} a) $x^2 + 2x + 1$; b) $4\beta^2 + 12\beta + 9$; c) $\omega^4 + 2\sqrt{2}\omega^2 + 2$; d) $\frac{4}{9} + \frac{4}{3}b + b^2$\\
\textbf{2.} a) $x^2 - 2x + 1$; b) $16\gamma^6 - 8\gamma^3 + 1$; c) $2\omega^2 - 2\sqrt{6}\omega + 3$; d) $\frac{1}{9} - \frac{2}{3}y + y^2$\\
\textbf{3.} a) $x^2 - 1$; b) $4\gamma^2 - 7$; c) $2\lambda^2 - 3$; d) $y^2 - \frac{4}{49}$\\
\textbf{4.} a) $(x + 3)^2$; b) $(a-1)^2 $; c) $(x^2 - 1)^2$; d) $(2b + 2)^2$; e) $(a^2 - 2)(a^2 + 2)$; f) $(2c + \sqrt{3})^2$; g) $\left(\frac{2}{3}z - \frac{1}{4}\right)^2$; h) $\left(\frac{g}{2} + 2\right)\left(\frac{g}{2} - 2\right)$; i) $\left(\frac{g}{2} - \sqrt{3}\right)\left(\frac{g}{2} + \sqrt{3}\right)$
}
\end{document}