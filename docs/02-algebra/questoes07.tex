\documentclass[a4paper,12pt]{article}
\usepackage[left=1.5cm, right=1.5cm, top=2cm,bottom=2cm]{geometry}
\usepackage{amsmath, amssymb, amsfonts}
\usepackage{multicol}
\usepackage{enumerate}
\usepackage{xcolor}

% https://tex.stackexchange.com/questions/29834/closed-square-root-symbol
\usepackage{letltxmacro}
\makeatletter
\let\oldr@@t\r@@t
\def\r@@t#1#2{%
\setbox0=\hbox{$\oldr@@t#1{#2\,}$}\dimen0=\ht0
\advance\dimen0-0.2\ht0
\setbox2=\hbox{\vrule height\ht0 depth -\dimen0}%
{\box0\lower0.4pt\box2}}
\LetLtxMacro{\oldsqrt}{\sqrt}
\renewcommand*{\sqrt}[2][\ ]{\oldsqrt[#1]{#2}}
\makeatother

\begin{document} 
  
\section*{Piape Matemática} 
\textbf{Módulo II - Malabarismos Algébricos}\\
\textbf{Exercícios Aula 07}         
\begin{multicols}{2}
\paragraph*{1.} Complete os quadrados nas seguintes expressões
\begin{enumerate}[a)]
\item $x^2 + 8x + 10$ 
\item $x^2 + 6x + 1$
\item $x^2 + 12x + 5$
\item $x^2 - 4x + 7$
\item $x^2 - 10x + 2$
\item $4x^2 + 8x - 3$
\item $4x^2 - 4x + 5$
\end{enumerate}

\paragraph*{2.} Resolva as equações completando os quadrados
\begin{enumerate}[a)]
\item $x^2 + 8x + 10 = 3$
\item $x^2 + 6x + 1 = 1$
\item $x^2 + 12x + 5 = -6$
\item $x^2 - 4x + 7 = 12$ %d 
\item $x^2 - 10x + 2 = 27$ %e
\item $4x^2 + 8x - 3 = 9$ %f
\item $4x^2 - 4x + 5 = 13$ %g
\end{enumerate}
\vspace*{0cm}
\end{multicols}
 
\vspace*{\fill}
{\footnotesize\color{darkgray}
\paragraph*{Gabarito} \hspace*{\fill}\\
\textbf{1.} a) $(x+4)^2 - 6$; b) $(x+3)^2 - 8$; c) $(x+6)^2 - 31$; d) $(x-2)^2 + 3$; e) $(x-5)^2 - 23$; f) $(2x+2)^2 - 7$; g) $(2x-1)^2 + 4$.
\textbf{2.} a) $x' = -7, x''=-1$; b) $x' = -6, x''=-0$; c) $x' = -11, x''=-1$; d) $x' = -1, x''=5$; e) $x = 5 \pm 5\sqrt{2}$ f) $x' = -3, x''=1$; g) $x' = -1, x''=2$.
}
\end{document}