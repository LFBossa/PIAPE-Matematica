\documentclass[a4paper,twocolumn,12pt]{article}
\usepackage[left=1.5cm, right=1.5cm, top=2cm,bottom=2cm]{geometry}
\usepackage{amsmath, amssymb, amsfonts}
\usepackage{enumerate}
\usepackage{xcolor}
\begin{document} 
  
\section*{Piape Matemática} 
\textbf{Módulo II - Malabarismos Algébricos}\\
\textbf{Exercícios Aula 01}         

\paragraph{1.} Calcule as seguintes potências
\begin{enumerate}[a)]
    \item $2^3$
    \item $-3^2$
    \item $(-5)^2$
    \item $10^4$
    \item $(-2)^4$
    \item $\left(\frac{1}{3}\right)^3$
    \item $\left(-\frac{2}{3}\right)^3$
    \item $0,\!5^3$
    \item $0,\!2^4$
\end{enumerate}

\paragraph*{2.} Calcule as seguintes potências de expoente negativo
\begin{enumerate}[a)]
    \item $2^{-3}$
    \item $-3^{-2}$
    \item $(-5)^{-2}$
    \item $10^{-4}$
    \item $(-2)^{-4}$
    \item $\left(\frac{1}{3}\right)^{-3}$
    \item $\left(-\frac{2}{3}\right)^{-3}$
    \item $0,\!5^{-3}$
    \item $0,\!2^{-4}$
\end{enumerate}

\newpage

\paragraph*{3.} Transforme as seguintes raízes em notação de potências
\begin{enumerate}[a)]
    \item $\sqrt{2}$
    \item $\sqrt[3]{3}$
    \item $\sqrt[4]{5^2}$
    \item $\sqrt[5]{7^{-3}}$
    \item $\sqrt[6]{11^{-5}}$
\end{enumerate}

\paragraph*{4.} Transforme as seguintes potências em notação de raízes
\begin{enumerate}[a)]
    \item $-2^{\frac{3}{2}}$
    \item $3^{-\frac{2}{3}}$
    \item $5^{\frac{3}{6}}$
    \item $\left(\frac{7}{4}\right)^{-\frac{3}{5}}$
    \item $0,\!4^{-\frac{5}{6}}$
\end{enumerate}
\vspace*{\fill}

{\footnotesize \color{darkgray}
\paragraph*{Gabarito} \hspace*{\fill}\\
\textbf{1.} a) 8; b) -9; c) 25; d) 10000; e) 16; f) $\frac{1}{27}$; g) $-\frac{8}{27}$; h) 0,125; i) 0,0016.\\
\textbf{2.} a) $\frac{1}{8}$; b) $-\frac{1}{9}$; c) $\frac{1}{25}$; d) $\frac{1}{10000}$; e) $\frac{1}{16}$; f) 27; g) $-\frac{27}{8}$; h) 8; i) 6250.\\
\textbf{3.} a) $2^{\frac{1}{2}}$; b) $3^{\frac{1}{3}}$; c) $5^{\frac{1}{2}}$; d) $7^{-\frac{3}{5}}$; e) $11^{-\frac{5}{6}}$.\\
\textbf{4.} a) $-\sqrt{2^3}$; b) $\frac{1}{\sqrt[3]{3^{2}}}$; c) $\sqrt{5}$; d) $\sqrt[5]{\left(\frac{4}{7}\right)^3}$; e) $\frac{1}{\sqrt[6]{0,\!4^5}}$.
}
\end{document}