\documentclass[a4paper,twocolumn,12pt]{article}
\usepackage[left=1.5cm, right=1.5cm, top=2cm,bottom=2cm]{geometry}
\usepackage{amsmath, amssymb, amsfonts}
\usepackage{enumerate}
\usepackage{graphicx}
\usepackage{tikz}
\begin{document} 
\section*{Piape Matemática} 
 
\subsection*{Módulo I}
\subsection*{Exercícios Aula 01}
 
\paragraph*{1.} Sejam as proposições $p$: ``O dia está chuvoso'' e $q$: ``O dia está frio''. Escreva as proposições compostas que representam as seguintes situações:
\begin{enumerate}[a)]
\item O dia está chuvoso e frio.
\item O dia não está chuvoso e está frio.
\item O dia está chuvoso ou frio.
\item O dia não está chuvoso nem frio.
\item O dia está chuvoso, mas não está frio.
\item O dia está frio, mas não está chuvoso.
\item Se o dia está chuvoso, então está frio.
\end{enumerate}
  
\paragraph*{2.} Calcule a tabela-verdade das seguintes proposições compostas:
\begin{enumerate}[a)]
\item $(p \land q) \to r$
\item $p \land (q\to r)$
\item $\neg p\lor q$
\end{enumerate}

\newpage

\paragraph*{3.} Sabendo que $p\to q$ é verdadeira, e que $q\to p$ é falsa, determine o valor lógico de $p$ e $q$.

\paragraph*{4.} Sabendo que $p\to q$ é verdadeira, e que $p\land q$ é falsa, determine o valor lógico de $p$ e $q$.

\vspace*{\fill}

{\footnotesize \color{darkgray}
\paragraph*{Gabarito} \hspace*{\fill}\\
\textbf{1.} a) $p\land q$; b) $\neg p\land q$; c) $p\lor q$; d) $\neg p\land \neg q$; e) $p\land \neg q$; f) $q \land \neg p$; g) $p\to q$.\\

\begin{minipage}{0.49\columnwidth}
  \textbf{2.} a)
  
  \begin{tabular}{|c|c|c|c|c|}
    \hline
    $p$ & $q$ & $r$ & $p \land q$ & (a) \\
    \hline
    V & V & V & V & V \\
    V & V & F & V & F \\
    V & F & V & F & V \\
    V & F & F & F & V \\
    F & V & V & F & V \\
    F & V & F & F & V \\
    F & F & V & F & V \\
    F & F & F & F & V \\
    \hline
    \end{tabular}
\end{minipage}\begin{minipage}{0.49\columnwidth}
  \textbf{2.} b)

 \begin{tabular}{|c|c|c|c|c|}
    \hline
    $p$ & $q$ & $r$ & $q \to r$ & (b) \\
    \hline
    V & V & V & V & V \\
    V & V & F & F & F \\
    V & F & V & V & V \\
    V & F & F & V & V \\
    F & V & V & V & F \\
    F & V & F & F & F \\
    F & F & V & V & F \\
    F & F & F & V & F \\
    \hline
    \end{tabular}
  \end{minipage}

  \begin{minipage}{0.49\columnwidth}
    \textbf{2.} c)
    
    \begin{tabular}{|c|c|c|c|}
      \hline
      $p$ & $q$ & $\neg p$ & $\neg p \lor q$ \\
      \hline
      V & V & F & V \\
      V & F & F & F \\
      F & V & V & V \\
      F & F & V & V \\
      \hline
      \end{tabular}
    
  \end{minipage}\begin{minipage}{0.49\columnwidth}

  \textbf{3.} $p$ é verdadeira e $q$ é falsa.\\
  \textbf{4.} $p$ é falsa e $q$ é verdadeira.
\end{minipage}
}


\end{document}