\documentclass[a4paper,twocolumn,12pt]{article}
\usepackage[left=1.5cm, right=1.5cm, top=2cm,bottom=2cm]{geometry}
\usepackage{amsmath, amssymb, amsfonts}
\usepackage{enumerate}
\begin{document} 
\section*{Piape Matemática} 
 
\subsection*{Módulo I}
 
\subsection*{Exercícios Aula 03}

\paragraph{1.} Para os exercício que segue, considere os seguintes conjuntos:
\begin{align*}
A &= \{1,2,3,4,8\}\\
B &=  \{2,4,6\}\\
C &= \{1,3,5\}
\end{align*}
%\[ A = \{1,2,3,4,8\}\] \[ B =  \{2,4,6\}\] \[ C = \{1,3,5\}\]
Calcule o que se pede. Represente o resultado em notação de diagramas de Venn.

\medskip

\begin{minipage}[t]{0.45\columnwidth}
  \begin{enumerate}[a)]
    \item \(A\cup B\)
    \item \(B\cup C\)
    \item \(A\cup B \cup C\)
    \item \(A\cap B\)
    \item \(B\cap C\)
  \end{enumerate}
\end{minipage}\begin{minipage}[t]{0.45\columnwidth}
  \begin{enumerate}[a)]
    \setcounter{enumi}{5}
    \item \(A\setminus B\)
    \item \(B\setminus A\)
    \item \(A\setminus C\)
    \item \(C\cap A\)
  \end{enumerate}
\end{minipage} 

\paragraph{2.} Calcule os tamanhos dos conjuntos
\begin{enumerate}[a)]
  \item \(|A\cup B|\)
  \item \(|A\cap B|\)
  \item \(|B\cup C|\)
  \item \(|B\cap C|\)
\end{enumerate}

Verifique que em todos os casos, vale a relação
\[|X\cup Y| = |X| + |Y| - |X\cap Y|\]


\paragraph{3.} Em uma turma, as pessoas ou praticam natação ou praticam vôlei. Sabe-se que 20 pessoas praticam natação, 15 praticam vôlei e 5 praticam ambos. Quantas pessoas há na turma?


\paragraph{4.} Uma turma possui 60 alunos. Destes, 40 praticam natação, 30 praticam vôlei e 20 praticam ambos. Quantos alunos não praticam nenhuma das duas atividades?

\paragraph{5.} Luciana está em um supermercado representando uma marca de café e, a cada cliente que aborda, ela oferece uma amostra grátis de três tipos de café (X, Y e Z). Após a degustação, o cliente responde a uma enquete a respeito dos tipos de café que gostou. Após coletar as respostas de 400 clientes, ela concluiu que:
\begin{itemize}
  \setlength{\itemsep}{1pt}
  \setlength{\itemindent}{1pt}
  \item 184 clientes gostaram do tipo X;
  \item 188 clientes gostaram do tipo Y;
  \item 220 clientes gostaram do tipo Z;
  \item 76 clientes gostaram dos tipos X e Y;
  \item 84 clientes gostaram dos tipos X e Z;
  \item 120 clientes gostaram dos tipos Y e Z; e,
  \item 52 clientes não gostaram de nenhum dos tipos X, Y e Z.
\end{itemize}
Considerando as informações, qual o número de clientes que gostaram dos três tipos de café? Utilize a fórmula 
\begin{multline*}
|A\cup B\cup C| = |A| + |B| + |C| \\ - |A\cap B| - |B\cap C| - |C\cap A| \\ + |A\cap B \cap C|
\end{multline*} 


\paragraph{6.} Essa é uma questão mais teórica: 
\begin{enumerate}[a)] 
\item Qual relação de inclusão entre os conjuntos \(A\), \(B\) e \(A\cup B\)?
\item Qual relação de inclusão entre os conjuntos \(A\), \(B\) e \(A\cap B\)?
\end{enumerate}


\vfill

{\footnotesize
\paragraph*{Gabarito} \hspace*{\fill}\\
\textbf{1.} a) \(A\cup B = \{1,2,3,4,6,8\}\); b) \(B\cup C = \{1,2,3,4,5,6\}\); c) \(A\cup B \cup C = \{1,2,3,4,5,6,8\}\); d) \(A\cap B = \{2,4\}\); e) \(B\cap C = \emptyset\); f) \(A\setminus B = \{1,3,8\}\); g) \(B\setminus A = \{6\}\); h) \(A\setminus C = \{2,4,8\}\); i) \(C\cap A = \{1,3\}\)\\
\textbf{2.} a) \(|A\cup B| = 6\); b) \(|A\cap B| = 2\); c) \(|B\cup C| = 6\); d) \(|B\cap C| = 0\)\\
\textbf{3.} 30 pessoas; \textbf{4.} 10 alunos; \textbf{5.} 36 clientes\\
\textbf{6.} a) \(A\subseteq A\cup B\) e também $B\subseteq A\cup B$; b) \(A\cap B\subseteq A\) e também \(A\cap B\subseteq B\)
}

\end{document}