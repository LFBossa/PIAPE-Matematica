\documentclass[a4paper,twocolumn,12pt]{article}
\usepackage[left=1.5cm, right=1.5cm, top=2cm,bottom=2cm]{geometry}
\usepackage{amsmath, amssymb, amsfonts}
\usepackage{enumerate}
\usepackage{graphicx}
\usepackage{xcolor}
\begin{document} 
\section*{Piape Matemática} 

\subsection*{Módulo I}
\subsection*{Exercícios Aula 04}

\paragraph*{1. } Vamos trabalhar em $\mathbb{Q}$, o conjunto dos números racionais. Calcule o que se pede:

\begin{enumerate}[a)]
\item $\frac{1}{3} + \frac{2}{5}$
\item $\frac{4}{7} - \frac{1}{3}$
\item $\frac{3}{4} + \frac{1}{2} - \frac{1}{8}$
\item $\frac{2}{7}\cdot \frac{3}{5}$
\item $-\frac{3}{4} \cdot \frac{1}{2}$
\item $\frac{3}{4} \div \frac{2}{3}$
\item $ \frac{7}{3} \div 5$
\end{enumerate}


\paragraph*{2. } Converta as frações em números decimais:
\begin{enumerate}[a)]
\item $-\frac{1}{7}$
\item $\frac{2}{5}$
\item $\frac{3}{4}$
\item $-\frac{5}{8}$
\end{enumerate}

\paragraph*{3. } Converta os números decimais em frações:
\begin{enumerate}[a)]
\item $3,\!14$
\item $3,\!1411\bar{1}$
\item $0,\!3333...$
\item $-2,\!5$
\item $-0,\!75$
\end{enumerate}


\paragraph*{4. } Racionalize o denominador das seguintes frações:
\begin{enumerate}[a)]
\item $\frac{1}{\sqrt{2}}$
\item $\frac{2}{\sqrt{3}}$
\item $\frac{-3}{\sqrt{5}}$
\item $\frac{2\sqrt{2}}{\sqrt{7}}$
\item $\frac{\sqrt{3}}{\sqrt{11}}$
\end{enumerate}

\paragraph*{5.} Calcule as seguintes operações, convertendo os números quando necessário:
\begin{enumerate}[a)]
\item $\frac{1}{2} + 0,\!3$
\item $\frac{1}{3} - 0,\!75$
\item $1,\!2\cdot\frac{5}{3} $
\item $0,\!5\div\frac{2}{3}$
\end{enumerate}
\vfill
% REMOVER OPERAÇÕES COM FRAÇÕES

{\footnotesize\color{darkgray}
\paragraph*{Gabarito} \hspace*{\fill}\\ 
\textbf{1.} a) $\frac{11}{15}$; b) $\frac{5}{21}$; c) $\frac{9}{8}$; d) $\frac{6}{35}$; e) $-\frac{3}{8}$; f) $\frac{9}{8}$; g) $\frac{7}{15}$\\
\textbf{2.} a) $-0,142857...$; b) $0,4$; c) $0,75$; d) $-0,625$\\
\textbf{3.} a) $\frac{314}{100}$; b) $\frac{31411}{10000}$; c) $\frac{1}{3}$; d) $-\frac{5}{2}$; e) $-\frac{3}{4}$\\
\textbf{4.} a) $\frac{\sqrt{2}}{2}$; b) $\frac{2\sqrt{3}}{3}$; c) $\frac{-3\sqrt{5}}{5}$; d) $\frac{2\sqrt{14}}{7}$; e) $\frac{\sqrt{33}}{11}$\\
\textbf{5.} a) $\frac{4}{5}$ ou $0,\!8$; b) $-\frac{5}{12}$ ou $-0,41666\ldots$; c) $2$; d) $\frac{3}{4}$
}
\end{document}