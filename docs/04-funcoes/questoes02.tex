\documentclass[a4paper,12pt]{article}
\usepackage[left=1.5cm, right=1.5cm, top=2cm,bottom=2cm]{geometry}
\usepackage{amsmath, amssymb, amsfonts}
\usepackage{multicol}
\usepackage{enumerate}

% https://tex.stackexchange.com/questions/29834/closed-square-root-symbol
\usepackage{letltxmacro}
\makeatletter
\let\oldr@@t\r@@t
\def\r@@t#1#2{%
\setbox0=\hbox{$\oldr@@t#1{#2\,}$}\dimen0=\ht0
\advance\dimen0-0.2\ht0
\setbox2=\hbox{\vrule height\ht0 depth -\dimen0}%
{\box0\lower0.4pt\box2}}
\LetLtxMacro{\oldsqrt}{\sqrt}
\renewcommand*{\sqrt}[2][\ ]{\oldsqrt[#1]{#2}}
\makeatother

\begin{document} 
  
\section*{Piape Matemática} 
\textbf{Módulo IV - Tudo é função}\\
\textbf{Exercícios Aula 02}         
\begin{multicols}{2}
\paragraph*{1.}  
Qual é a notação das seguintes funções de $\mathbb{R}$ em $\mathbb{R}$?
\begin{enumerate}[a)] 
    \item $f$ associa cada número real ao seu oposto.
    \item $g$ associa cada número real ao seu cubo.
    \item h associa cada número real ao seu quadrado menos 1.
    \item k associa cada número real ao número 2.
\end{enumerate}

Para as questões 2 e 3 a seguir, definimos as seguintes funções
$$
\begin{aligned} 
f:\mathbb{R} &\to \mathbb{R} & g:\mathbb{R}_{+} & \to \mathbb{R} \\
x &\mapsto 2x + 1 & x&\mapsto \sqrt{2x} + 2\\[1em]
h: \mathbb{N} &\to \mathbb{Z} &   &  \\
n &\mapsto 2n^2 - 5 &   &  \\
\end{aligned}
$$
\paragraph*{2.} Calcule os valores pedidos:
\begin{multicols*}{2}
    \begin{enumerate}[a)]
        \item $f(3)$ % 7
        \item $g(2)$ % 4
        \item $h(5)$ % 45
        \item $f\left(\frac{1}{2}\right)$ % 2
        \item $g\left(\frac{9}{2}\right)$ % 5
        \item $h(0)$ % -5
        \item $f(-1.7)$ % -2.4
        \item $g(3.5)$ % \sqrt{7} + 2
        \item $h(4)$ % 27
        \item $f\left(\sqrt{2}\right)$ % 2\sqrt{2} + 1
        \item $g\left(\sqrt{12}\right)$ % 2\sqrt[4]{3} + 2
    \end{enumerate}
    
\end{multicols*}
 
\paragraph*{3.} Explique porquê não podemos calcular: 
    \begin{enumerate}[a)]
        \item $g(-1)$
        \item $h\left(\frac{1}{2}\right) $
    \end{enumerate}
\end{multicols}
 

\vspace*{\fill}
{\footnotesize
\paragraph*{Gabarito} \hspace*{\fill}\\
\textbf{1.} a) $f(x) = -x$; b) $g(x) = x^3$; c) $h(x) = x^2 - 1$; d) $k(x) = 2$.\\
\textbf{2.} a) 7; b) $4$; c) 45; d) 2; e) $5$; f) -5; g) -2.4; h) $\sqrt{7} + 2$; i) 27; j) $2\sqrt{2} + 1$; k) $2\sqrt[4]{3} + 2$.\\
\textbf{3.} a) $g(-1)$ não está definido pois $-1 \notin \mathbb{R}_{+}$; b) $h\left(\frac{1}{2}\right)$ não está definido pois $\frac{1}{2} \notin \mathbb{N}$.
}
\end{document}