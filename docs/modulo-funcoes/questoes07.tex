\documentclass[a4paper,12pt]{article}
\usepackage[left=1.5cm, right=1.5cm, top=2cm,bottom=1cm]{geometry}
\usepackage{amsmath, amssymb, amsfonts, amsthm}
\usepackage{multicol}
\usepackage{enumerate}
\usepackage{pgfplots}
\usepackage{tikz}

% https://tex.stackexchange.com/questions/29834/closed-square-root-symbol
\usepackage{letltxmacro}
\makeatletter
\let\oldr@@t\r@@t
\def\r@@t#1#2{%
\setbox0=\hbox{$\oldr@@t#1{#2\,}$}\dimen0=\ht0
\advance\dimen0-0.2\ht0
\setbox2=\hbox{\vrule height\ht0 depth -\dimen0}%
{\box0\lower0.4pt\box2}}
\LetLtxMacro{\oldsqrt}{\sqrt}
\renewcommand*{\sqrt}[2][\ ]{\oldsqrt[#1]{#2}}
\makeatother

\newcommand{\novografico}[3]{
\begin{tikzpicture}
    \begin{axis}[
        grid=both,
        grid style={line width=.1pt, draw=gray!50, dashed},
        major grid style={line width=.2pt,draw=gray!50, dashed},
        axis lines=middle,
        enlargelimits,
        samples=75,
        domain=#1,
        width=5cm,
        height=5cm,
        xlabel=$x$,
        ylabel={$#3$},
        xlabel style={at={(current axis.right of origin)}, anchor=west},
        ylabel style={at={(current axis.above origin)}, anchor=south},
        unbounded coords=jump
    ]
    \addplot[blue, thick] {#2}; 
    \end{axis}
\end{tikzpicture}
}

\begin{document} 
  
\section*{Piape Matemática} 
\textbf{Módulo IV - Tudo é função}\\
\textbf{Exercícios Aula 07}    

\begin{multicols}{2} 
 
\paragraph*{1.}  Encontre as funções inversas das funções dadas:

\begin{enumerate}[a)]
\item $f(x) = 2x+3$
\item $g(x) = 4 + 1/x$
\item $h(x) = x^2 + 3$
\item $i(x) = x^3$ 
\item $j(x) = x^2 + 2x + 3$
\item $k(x) = \dfrac{2x + 3}{x - 5}$
\end{enumerate}

\paragraph*{2.} Considere a função $f$ determinada pela tabela a seguir. Complete a tabela da função inversa $f^{-1}$, ordenando os valores de $x$ em ordem crescente.
\begin{center}
\begin{tabular}{|c|c|}
\hline
$x$ & $f(x)$ \\
\hline
1 & -2 \\
2 & 3 \\
3 & $\frac{3}{4}$ \\
4 & $\sqrt{5}$ \\
5 & $-\frac{1}{3}$ \\
\hline
\end{tabular}\hspace*{16mm}
\begin{tabular}{|c|c|}
    \hline
$x$ & $f^{-1}(x)$ \\
\hline
    & \\
    & \\
    & \\
    & \\
    & \\
\hline
\end{tabular}
\end{center}


\paragraph*{3.} Considere $a\in\mathbb{R}$ um número real qualquer, e seja $$f(x) = \frac{a}{x}$$

Verifique que $f^{-1}(x) = f(x)$.

\end{multicols}

\vspace*{\fill}
{\footnotesize
\paragraph*{Gabarito} \hspace*{\fill}\\
\textbf{1.} a) $f^{-1}(x) = \frac{x-3}{2}$; b) $g^{-1}(x) = \frac{1}{x-4}$; c) $h^{-1}(x) = \sqrt{x-3}$; d) $i^{-1}(x) = \sqrt[3]{x}$; e) $j^{-1}(x) = \sqrt{x-2} - 1 $; f) $k^{-1}(x) = \frac{5x + 3}{x - 2}$.
}
\end{document}