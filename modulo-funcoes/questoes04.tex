\documentclass[a4paper,12pt]{article}
\usepackage[left=1.5cm, right=1.5cm, top=2cm,bottom=2cm]{geometry}
\usepackage{amsmath, amssymb, amsfonts, amsthm}
\usepackage{multicol}
\usepackage{enumerate}

% https://tex.stackexchange.com/questions/29834/closed-square-root-symbol
\usepackage{letltxmacro}
\makeatletter
\let\oldr@@t\r@@t
\def\r@@t#1#2{%
\setbox0=\hbox{$\oldr@@t#1{#2\,}$}\dimen0=\ht0
\advance\dimen0-0.2\ht0
\setbox2=\hbox{\vrule height\ht0 depth -\dimen0}%
{\box0\lower0.4pt\box2}}
\LetLtxMacro{\oldsqrt}{\sqrt}
\renewcommand*{\sqrt}[2][\ ]{\oldsqrt[#1]{#2}}
\makeatother

\begin{document} 
  
\section*{Piape Matemática} 
\textbf{Módulo IV - Tudo é função}\\
\textbf{Exercícios Aula 04}    

\begin{multicols}{2}
\paragraph*{1.} Considere as seguintes funções de $\mathbb{R}$ em $\mathbb{R}$:
\begin{align*}
f(x) &= x + \frac{2}{x}\\
g(x) &= x^2 + 1\\
h(x) &= \sqrt{2x}\\
\end{align*} 
Calcule as composições que são pedidas:
\begin{enumerate}[a)] 
    \item $f\circ g$
    \item $f\circ h$
    \item $g\circ f$
    \item $h\circ g$
    \item $f\circ f$
    \item $g\circ g$
    \item $h\circ f$
\end{enumerate}

\paragraph*{2.} Vamos fazer a `decomposição' das expressões abaixo: encontre $f$ e $g$ tais que $f\circ g$ seja igual as expressões abaixo.\\
({\footnotesize Não vale usar a função $f(x)  = x$.})
\begin{enumerate}[a)]  
    \item $x^2 + 3$
    \item $\sqrt{2x} + 1$
    \item $x^2 + \frac{1}{x^2}$
    \item $\frac{1}{x^2 + 1}$
    \item $e^{2x + 1}$
    \item $\cos(2x+2)$
\end{enumerate}

\vspace*{5cm}
\end{multicols}

\vspace*{\fill}
{\footnotesize
\paragraph*{Gabarito} \hspace*{\fill}\\
\textbf{1.}   
a) $f \circ g  = x^{2} + 1 + \frac{2}{x^{2} + 1}$;
b) $f \circ h  = \sqrt{2x} + \frac{\sqrt{2}}{\sqrt{x}}$;
c) $g \circ f  = x^2 + \frac{4}{x^2} + 5 $;
d) $h \circ g  = \sqrt{2x^{2} + 2}$;
e) $f \circ f  =x + \frac{2}{x} + \frac{2x}{x^2+2}$
f) $g \circ g  = x^4 + 2x^2 + 2$
g) $h \circ f  = \sqrt{2 x + \frac{4}{x}}$\\
\textbf{2.} Essas são apenas algumas sugestões de decomposição, outras respostas são possíveis.\\
a) $f(x) = x + 3$ e $g(x) = x^2$;
b) $f(x) = \sqrt{x} + 1$ e $g(x) = 2x$;
c) $f(x) = x + \frac{1}{x}$ e $g(x) = x^2$;
d) $f(x) = \frac{1}{x}$ e $g(x) = x^2 + 1$;
e) $f(x) = e^x$ e $g(x) = 2x + 1$;
f) $f(x) = \cos(x)$ e $g(x) = 2x + 2$.
}
\end{document}