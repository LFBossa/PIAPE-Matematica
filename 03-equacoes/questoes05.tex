\documentclass[a4paper,12pt]{article}
\usepackage[left=1.5cm, right=1.5cm, top=2cm,bottom=2cm]{geometry}
\usepackage{amsmath, amssymb, amsfonts}
\usepackage{multicol}
\usepackage{enumerate}

% https://tex.stackexchange.com/questions/29834/closed-square-root-symbol
\usepackage{letltxmacro}
\makeatletter
\let\oldr@@t\r@@t
\def\r@@t#1#2{%
\setbox0=\hbox{$\oldr@@t#1{#2\,}$}\dimen0=\ht0
\advance\dimen0-0.2\ht0
\setbox2=\hbox{\vrule height\ht0 depth -\dimen0}%
{\box0\lower0.4pt\box2}}
\LetLtxMacro{\oldsqrt}{\sqrt}
\renewcommand*{\sqrt}[2][\ ]{\oldsqrt[#1]{#2}}
\makeatother

\begin{document} 
  
\section*{Piape Matemática} 
\textbf{Módulo III - (In)equações}\\
\textbf{Exercícios Aula 05}         
\begin{multicols}{2}
\paragraph*{1.} Resolva as equações irracionais:
\begin{enumerate}[a)]     
    \item $\sqrt{3x-2} = 4$ % 6
    \item $\sqrt{x^2-5x+13} = 3$ % 1, 4
    \item $\sqrt{5x+10} = 17 - 4x $ %3
    \item $x + \sqrt{25 - x^2} = 7$ % 3, 4
    \item $\sqrt{x^2 + x -1} = 2 - x$ % 1
    \item $x - 5\sqrt{x} + 6  = 0$ % 4, 9
    \item $\sqrt{x} + 1 = \sqrt{2x+1}$ % 0, 4
    \item $x -\sqrt{x} - 12 = 0$ % 16
    \item $\sqrt[3]{2x+5} = -3$ % -16
    \item $\sqrt[3]{x^2 - x - 4} = 2$ % 4, -3
\end{enumerate}
 
 \vspace*{2cm}
\end{multicols}
 
\vspace*{\fill}
{\footnotesize
\paragraph*{Gabarito} \hspace*{\fill}\\
\textbf{1.}
    a)  $x = 6$ 
    b)  $x = 1$ ou $x = 4$
    c)  $x = 3$
    d)  $x = 3$ ou $x = 4$ 
    e)  $x = 1$ 
    f)  $x = 4$ ou $x = 9$ 
    g)  $x = 0$ ou $x = 4$
    h)  $x = 16$  
    i)  $x = -16$
    j)  $x = 4$ ou $x = -3$\\
\end{document}