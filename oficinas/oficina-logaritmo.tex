\documentclass[a4paper,12pt]{article}
\usepackage[left=1.5cm, right=1.5cm, top=2cm,bottom=1cm]{geometry}
\usepackage{amsmath, amssymb, amsfonts, amsthm}
\usepackage{multicol}
\usepackage{enumerate}
\usepackage{pgfplots}
\usepackage{tikz}

% https://tex.stackexchange.com/questions/29834/closed-square-root-symbol
\usepackage{letltxmacro}
\makeatletter
\let\oldr@@t\r@@t
\def\r@@t#1#2{%
\setbox0=\hbox{$\oldr@@t#1{#2\,}$}\dimen0=\ht0
\advance\dimen0-0.2\ht0
\setbox2=\hbox{\vrule height\ht0 depth -\dimen0}%
{\box0\lower0.4pt\box2}}
\LetLtxMacro{\oldsqrt}{\sqrt}
\renewcommand*{\sqrt}[2][\ ]{\oldsqrt[#1]{#2}}
\makeatother

\newcommand{\novografico}[3]{
\begin{tikzpicture}
    \begin{axis}[
        grid=both,
        grid style={line width=.1pt, draw=gray!50, dashed},
        major grid style={line width=.2pt,draw=gray!50, dashed},
        axis lines=middle,
        enlargelimits,
        samples=75,
        domain=#1,
        width=5cm,
        height=5cm,
        xlabel=$x$,
        ylabel={$#3$},
        xlabel style={at={(current axis.right of origin)}, anchor=west},
        ylabel style={at={(current axis.above origin)}, anchor=south},
        unbounded coords=jump
    ]
    \addplot[blue, thick] {#2}; 
    \end{axis}
\end{tikzpicture}
}

\begin{document} 
  
\section*{Piape Matemática} 
\textbf{Oficina III - Logaritmos}    

\begin{multicols}{2} 
 
\paragraph*{1.} Usando a tabela, calcule os seguintes Logaritmos decimais:
\begin{enumerate}[a)]
\item $\log(5)$
\item  $\log(31)$
\item $\log(80)$
\item $\log(5,\!2)$
\item $\log(31,\!4)$
\item $\log(288)$
\item $\log(3000)$
\item $\log(2650)$
\item $\log(0,\!006)$
\item $\log(0,\!000741)$
\item $\log(1,\!63\cdot 10^{5})$
\item $\log(6,\!53\cdot 10^{-6})$
\end{enumerate}

\paragraph*{2.}  Usando interpolação, calcule os seguintes logaritmos decimais:
\begin{enumerate}[a)]
\item $\log(7552)$
\item $\log(3,\!2525)$
\item $\log(0,\!09366)$
\end{enumerate}

\paragraph*{3.} Utilize a propriedade dos logaritmos para efetuar as seguintes operações:

\begin{enumerate}[a)]
    \item $230\cdot 345$
    \item $0,\!043\cdot 0,\!0054$
    \item $456\cdot 0,\!339$
    \item $78500\cdot 0,\!00062$
    \item $1,65\cdot 10^{4}\cdot 2,3\cdot 10^{3}$
    \item $853\div 225$
    \item $0,\!00045\div 0,\!00015$
    \item $50200\div 0,\!00025$
    \item $7,4\cdot 10^{5}\div 2,2\cdot 10^{2}$
    \item $235^7$
    \item $0,\!0003^5$
\end{enumerate}


\paragraph*{4.} Calcule os seguintes logaritmos usando a mudança de base:

\begin{enumerate}[a)]
    \item $\log_{3}(5)$
    \item $\log_{5}(3)$
    \item $\log_{7}(2)$
    \item $\log_{2}(7)$ 
\end{enumerate}
\end{multicols}


\vspace*{\fill}
{\footnotesize
\paragraph*{Gabarito} \hspace*{\fill}\\
\textbf{1.}  a) 0,6990; b) 1,4914; c) 1,9031; d) 0,7160; e) 1,4949; f) 2,4594; g) 3,4771; h) 3,4232; i) -2,2218; j) -3,1302; k) 5,2122; l) -5,1851\\
\textbf{2.} a) 3,87802; 
b) 0,512225;
c) -1,02846;
\\
\textbf{3.} a) 79350; b) 0,0002322; c) 154,224; d) 48,7; e) $3,795\cdot 10^{7}$; f) 3,79; g) 3; h) 200800; i) 3,3636; j) 1,5$\cdot 10^{14}$; k) 2,187$\cdot 10^{-15}$\\
\textbf{4.} a) 1,48; b) 0,676; c) 0,356; d) 2,81
}
\end{document}